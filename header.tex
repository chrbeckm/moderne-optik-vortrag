% Load packages you need here
\usepackage{polyglossia}
\setmainlanguage{german}

\usepackage{csquotes}
\usepackage{scrhack}
%\usepackage[aux]{rerunfilecheck}
%\usepackage{fontspec}
%\recalctypearea{}


\usepackage{amsmath}
\usepackage{amssymb}
\usepackage{mathtools}

\usepackage[
  math-style=ISO,
  bold-style=ISO,
  sans-style=italic,
  nabla=upright,
  partial=upright,
  warnings-off={
    mathtools-colon,
    mathtools-overbracket,
  },
]{unicode-math}
\usepackage[
  locale=DE,
  separate-uncertainty=true,
  per-mode=symbol-or-fraction,
]{siunitx}
\usepackage{xfrac}
\usepackage{float}
\floatplacement{figure}{htbp}
\floatplacement{table}{htbp}
\usepackage{pgfplotstable}
\usepackage{array}
\usepackage{subcaption}
\usepackage{graphicx}
\usepackage{booktabs}
\usepackage{microtype}


\usepackage{hyperref}
\usepackage{bookmark}

\usepackage[backend=biber]{biblatex}
\addbibresource{lit.bib}
% load the theme after all packages

\usetheme[
  showtotalframes, % show total number of frames in the footline
]{tudo}

% Put settings here, like
\unimathsetup{
  math-style=ISO,
  bold-style=ISO,
  nabla=upright,
  partial=upright,
  mathrm=sym,
}
